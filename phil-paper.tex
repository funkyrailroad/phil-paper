%%
%% This is file `lpstemplate.tex',
%% generated with the docstrip utility.
%%
%% The original source files were:
%%
%% lps.dtx  (with options: `template')
%% 
%%  Copyright 2005, 2006 Gustavo Cevolani, Gruppo Utilizzatori Italiani di TeX
%% 
%%  This program is provided under the terms of the
%%  LaTeX Project Public License distributed from CTAN
%%  archives in directory macros/latex/base/lppl.txt.
%% 
%%  Author: Gustavo Cevolani
%%          at g.cevolani@gmail.com
%% 
%%  This work has the LPPL maintenance status "author-maintained".
%% 
%\documentclass{lps}
\documentclass[a4paper]{article}
%%
%% User's packages:
%% \usepackage[italian]{babel}          % for Italian language
%% \usepackage[ansinew]{inputenc}   % to insert special symbols on Windows systems
%% \usepackage[T1]{fontenc}            % standard encoding on Windows systems
\usepackage{lipsum}                   % to test this template

\begin{document}
%% Article's information:
\title{Title of the article}
%% Author's information:
\author{John First\\Department of Philosophy\\University of Trieste (Italy)\\j.first@mail.it \and%
            James Second\\Department of Philosophy\\University of Trieste (Italy)\\j.second@mail.it}
%\shortauthor{John First, James Second}
%% Issue's information:
%%\Volume{}
%%\Number{}
%%\Year{}
%%\FirstPage{}
%%

\maketitle
\newpage

Research Question: What is inclusive foreign aid, is it a transdisciplinary field, and what
insights are offered from a philosophy of science perspective?

\tableofcontents

%% Keywords and abstract:
%\keywords{some, crucial, terms.}


\begin{abstract}
Summary of the article.
\end{abstract}
\newpage
%%

%%%%%%%%%%%%%%%%%%%%%%%% Main text: %%%%%%%%%%%%%%%%%%%%%%%%



\section{What is inclusive foreign aid?}

\subsection{A brief introduction on the subject, some historical development}

Inclusive foreign aid is foreign aid or development work that is aimed
primarily at advancing the status and welfare of people with disabilities
(PWDs) in underdeveloped countries, particularly the Least Developed Countries
(LDCs). LDC is a classification used by the UN which indicates a certain level
of per capita income of a country. Of the approximately 1300 million people
who survive on less than 1 USD per day, over 200 million of them have some
form of disability or impairment. PWDs face additional difficulties when
compared to their non-impaired brothers and sisters, and for this reason
additional considerations are necessary if PWDs are to benefit from typical
development aid projects. 


\subsection{What are some of the recognized problems in this field?}

\subsubsection{No access to health facilities}

Obstacles faced by PWDs in LDCs are numerous. Immediately after an accident
especially, but also in the case of standard medical check-ups, PWDs need
access to health facilities in order to be given proper treatment if needed.
Many health facilities are not able to accommodate PWDs. This could be in the
form of barriers to entry, which could include a lack of elevator, only stairs
outside and no ramp. However this can also mean that hospitals simply to not
have the equipment to assist or rehabilitate those with disabilities. For
example, if a hospital does not have any crutches, prosthetic legs or
wheelchairs, then that hospital is not able to help a person who just lost
their leg become more mobile. 


\subsubsection{No/fewer employment opportunities}

PWDs in general have a more difficult time finding or maintaining employment.
At the very least, having a disability limits the range of possible employment
opportunities. If a person becomes disabled, the combination of their
disability and their employment will decided whether or not the person will be
able to keep their. Chowdhury et al. 2006 observe that a shopkeeper and farmer
who sustain an injury that leads to the same disability will have immensely
different prospects depending on the disability. In the case of a missing leg,
the shopkeeper will largely be able to continue his daily work as it is
largely sedentary, whereas the farmer will suffer far worse consequences and
likely no longer be able to continue.

\subsubsection{Poor access to credit}

A further difficulty faced by PWDs in LDCs are a lack of opportunities for
credit. This is not only true for PWDs, but generally for impoverished
individuals in LDCs, however there has been a link established between poverty
and disability. This aspect is causally related to the no/fewer employment
opportunities aspect because it limits any entrepreneurial activity that might
be undertaken. The lack of legitimate credit options sets the scene for
predatory credit sharks to take advantage of those in need. Additionally,
merely the availability of access to credit does not solve the problem of the
loan shark; credit must be available to those in need, and its availability
must be made known to those in need.


\subsubsection{Illiteracy}

A final obstacle faced by the world’s poor and disabled is a lack of basic
education. The reasons can be myriad. Perhaps there are no facilities nearby
that are capable of providing such education, perhaps the facilities are not
equipped such that PWDs might benefit from them, or perhaps the PWD is
mentally deficient in some respect and unable to progress in an educational
environment suited to more normally-abled people. Regardless of the reason,
the effects are the same.


\subsection{The causes of impairment range }

The types of disability from lifelong conditions to something caused in an
accident of some sort. Lifelong conditions can be blindness, deafness,
muteness, lame limbs, etc. Disability caused by an accident can be due to
uncleared minefields, an amputation, trauma, etc. The cause of a specific
impairment does not matter as much as what the actual impairment is because
solutions address the impairment itself as opposed to the cause. 


\subsection{What are some of the proposed solutions in this field?}

\subsubsection{Microfinance opportunities}

In line with the social model's definition of independence, another way to
give PWDs more autonomy is to give them credit or at least provide them with
credit opportunities. This credit can be seen a way to enable PWDs to make
their own decisions and be responsible for the repercussions of this. Because
PWDs in LDCs are very poor (as much of the rest of the population in LDCs),
even very small loans are able to make a large difference. Typical in the case
of an impoverished population is a self-perpetuating cycle of poverty. A poor
person doesn't have any money, so they need to borrow the supplies and the
materials to make a good which they can later sell, but they borrow the
supplies and raw materials at a price that can mean they barely break even at
the end of it.


\section{How is knowledge generated in this field?}


This field is currently an active field, and accordingly there are various
research efforts underway generating knowledge. This then begs the question,
how is knowledge generated in this field?  

\subsection{What is knowledge?}

Before we get too far ahead of ourselves, one must first answer ``What is
knowledge?'' and ``What does it mean to know?'' One conception of knowledge
envisions it as a justified true belief (JTB). It is possible to know many
things (like how to ride a bike or a particular person) but this definition
focuses on propositions of the form, ``How does S know p?'' where S is a subject
and p is some proposition like ``The Earth is round.'' In order for p to be
known, p must be true. This is axiomatic in the definition of knowledge
because it does not do much good to have false knowledge. In addition to p
being true, S must believe p. To know something is a stricter condition that
to believe something, so in order to know, one must also believe.
Additionally, S must have proper justification for believing that p. Without
justification, one could merely accidentally believe a true p. Each aspect is
necessary for knowledge, and combined they are sufficient for knowledge.

However, this meaning of 

\subsection{The contextual distinction: discovery and justification}

One model of knowledge generation involves a dichotomy of contexts: the
context of discovery and the context of justification. The context of
discovery involves the efforts exerted while trying to get a grasp on the
topic at hand. This can involve data collection and building hypotheses. The
context of justification on the other hand involves providing support for a
given hypothesis or theory. The latter is more clearly within the domain of
philosophy of science; it involves verifying if what we think is true is
indeed true, which is very much related to epistemological concern of ``How do
we know what we know?''

\subsubsection{Context of discovery: epistemological methods used in this field}

\paragraph{Participatory research}

Participatory research is a way of gathering information from a target
population in a way that enables the target population to contribute to and
shape the knowledge gleaned from the interaction. As an example, two
anthropologists observing an indigenous tribe can sit quietly and observe many
situations without disturbing (or ``interacting with,'' to use a word with a
more neutral connotation) the objects of their observation. This methodology
is limited in that it only allows for knowledge to be generated from the
perspective/paradigm of the anthropologists, and the perspective/paradigm of
the indigenous tribe is completely ignored (or unutilized, to again use a word
with a more neutral connotation.)

Participatory research intends to incorporate this additional perspective into
the knowledge-generation process. In the above example, the anthropologists
make their observations but then would continue to engage in a dialogue with
the indigenous people. In this way, the biases introduced by the
anthropologists via their inductive generalizations are subjected to the input
and criticism of the natives. This introduction of a new viewpoint allows for
the ``averaging of viewpoints'' process that is outlined by Datson. According
to Datson, this process is the road to aperspectival objectivity.
Aperspectival objectivity is objectivity which loses the individual
idiosyncrasies involved in the knowledge-generation process, i.e. the biases
and preconceived notions held by the observers. Aperspectival objectivity has
also been disparagingly dubbed ``the view from nowhere,'' since the
elimination of idiosyncrasies can be said to amount to the elimination of the
individual perspective. However, this author finds a more fitting metaphor for
aperspectival objectivity to be ``the view from everywhere.`` The averaging
process outlined by Datson does not involve eliminating old elements, but
rather incorporating new elements (i.e.  perspectives), and for this reason
should be seen as an additive operation.  The typical mathematical operation
associated with averaging is the arithmetic mean. In this process, N different
values, which can represent anything, are all multiplied by the same fraction
1/N and then summed. In this case again, averaging is an additive process.


\section{What is a transdisciplinary field?}

The concept of transdisciplinarity is a subtopic in the field of
interdisciplinarity. To get an idea of exactly what transdisciplinarity is,
it’s helpful to have an overview of interdisciplinarity and what that means.

\subsection{What are disciplinarity and disciplines?}

Interdisciplinarity stands in contrast to the standard disciplines with which
those of us in modern western universities are very familiar with. To name a
few, they include classic subjects like physics, chemistry, economics,
psychology, etc. Although seemingly set in stone in the modern university,
these disciplines evolved into their current form over the course of the last
300 years. As is usual with historical processes, their evolution is not
entirely predetermined and if it were to all happen again, there stands the
chance that the disciplines would have emerged in a different form. Given the
arbitrariness involved, it leads one to wonder what other disciplines could
have emerged given certain other circumstances. One can envision a merging or
fracturing of the various modern disciplines that might be more directed at
solving slightly or potentially even drastically different problems. 

\subsection{What is multidisciplinarity?}

The first step on the way to transdisciplinarity from standard academic
disciplines is multidisciplinarity. Multidisciplinarity involves the
perspective of multiple disciplines. In doing this, the multidisciplinary
approach provides access to more knowledge and methods than a traditional
disciplinary approach, but each discipline remains fully intact. As a result,
there is a lack of integration of the different perspectives. Words like
juxtaposing, sequencing, and coordinating fit to the multidisciplinary
approach.

\subsection{What are interdisciplinarity and interdisciplines?}

The next step is known as interdisciplinarity. The primary difference is the
amount of integration of the individual disciplinary perspectives that is
achieved. Words like interacting, integrating, focusing, blending, and linking
aptly describe this approach. An interdisciplinary approach can be the result
of borrowing the methods of a different discipline, although the degree to
which this impacts the borrowing discipline can vary. A stronger or more
``genuine'' form of interdisciplinarity can be the result of theoretical
integration. In this sense of interdisciplinarity, one overarching theoretical
framework can offer insight into multiple disciplines, and findings in one
discipline can contribute to the problems and theories of another.


\subsection{What are transdisciplinarity and transdisciplines?}

Finally we arrive at transdisciplinarity. Transdisciplinarity, in the sense
relevant to this work, focuses on the generation of knowledge directed at
solving real-world problems. This amounts to letting the real-world and its
inhabitants dictate which problems are important and in need of solving as
opposed to the traditional academic disciplinary structures. Because of the
potential divergence between the needs of a specific real-world problem and
the specialization of a traditional discipline, it is possible that the
knowledge and methods needed to address a given real-world problem are spread
across different disciplines. This is a familiar concept to
interdisciplinarity as explained above, with the added stipulation that some
external (to academia) stakeholders also contribute to the knowledge
generation process. As a result, transdisciplinarity can be thought of as
interdisciplinarity with societal input. 

\section{How is inclusive foreign aid a transdisciplinary field?}

Inclusive foreign aid is a transdisciplinary field because it both addresses a
societally relevant problem and uses an interdisciplinary approach to
addressing this problem. 

\subsection{Extra-academic genesis of the problem}

\subsubsection{Problem Framing}

The genesis of this problem clearly does not lie in any of the traditional
disciplines. The entire concept of foreign aid is focused on solving
real-world problems. The issue of who exactly are the stakeholders is however
an interesting one. The advances in disability studies in the USA make it
clear that the stakeholders should be the ones affected by the suggested
policies and solutions with the mantra ``Nothing about us, without us.''



\subsection{Which disciplines are involved and how do they interact in an
interdisciplinary manner?}

\paragraph{What types of things are drawn on from the different disciplines?}

Inclusive foreign aid includes aspects from a variety of different disciplines
including disability studies, finance, education, medicine, psychology,
sociology, bionics, and engineering. 

\paragraph{How are these things integrated?}

There is however a difference between calling upon the knowledge and
methodologies of those disciplines in a merely juxtapositional manner and
doing so in a more integrative manner (i.e. the multidisciplinarity or
interdisciplinarity as explained above).

\section{Inclusive foreign aid from the perspective of various philosophers of
science}

\subsection{Kuhnian Perspective}

\subsubsection{Establishment of a Paradigm}

There are a number of paradigms within disability studies, and also a number
of problems that the field is oriented towards solving. The different
paradigms represent different approaches to solving these problems. The first
conceptualization of disability is known as the medical model. It views a
person with a disability in a way similar to a sick person. The disability is
viewed as a deficiency that may be potentially remedied via medical
intervention so that the person may be able to live as close to a ``normal''
life as possible. 

A second paradigm that came as a reaction to the medical model is called the
social model. It offers a social constructionist view of the situation whereby
the person with disability (in the sense of the medical model) is only
rendered  disabled by the societal conditions and norms. The social model
offers a distinction between an impairment and a disability. An impairment can
be defined as ``the condition of a person lacking part or all of limb, organ
or mechanism of the body.'' A mechanism of the body is to be understood as a
sense like hearing or sight, but can also include something more advanced like
the ability to read. Disability is then defined as ``the disadvantage or
restriction of activity caused by a contemporary social organization which
\ldots excludes [people with physical impairments] from participation in the
mainstream of social activities.'' In this sense, an impairment does not mean a
person is disabled if they are not excluded in some sense. For example, a
person without legs is undeniably impaired, but the state their disability is
defined to be context-specific. If this person is in a building without an
elevator and needs to gain access to the roof, he or she is disabled. However,
build an elevator and the disability disappears. Conversely, with this set of
definitions it is also possible for a person to have a disability without
being impaired. An example of this is a left-handed person who has difficulty
using tools that are traditionally built for use by a right-handed person.

An additional way of seeing the situation of a person with disability (PWD) is
to posit that they are being actively oppressed (whether intentionally or not)
by people without disabilities. In essence, that people without disabilities
are denying PWDs certain basic social rights. This is seen as an intolerable
injustice, and the question becomes what dominant beliefs and practices of the
time enable this. There are two basic premises of the social model (Winter
2003): 1) An impairment is made into a disability only by social conditions
and 2) Any policy should focus on giving the most autonomy to the impaired
person as possible.  



\subsection{Latourian perspective}


\subsubsection{Quasi-objects in Inclusive Foreign Aid}

\paragraph{What is a quasi-object}

Bruno Latour outlines the idea of a Quasi-object in his book ``We Have Never
Been Modern.'' He presents a problem with the traditional subject/society and
object/nature dichotomy. There are two faulty beliefs refuted by contradictory
denunciations when it comes to the interaction between objects and subjects. 

Belief 1: Objects have an intrinsic value, like money, gods, and status. 

Denunciation 1: Those things are intrinsically worthless and society merely
projects its values onto these objects, which are essentially blank slates.

In this view of things, the objects are weak and worthless, and society is
powerful enough to be capable of making anything out of them.

Belief 2: Subjects have free will and have the power to do whatever it is they
like.

Denunciation 2: This isn't true, there are natural constraints on subjects
that limit what they can do.  In this view of things, subjects are weak over
the influence of an overwhelmingly powerful nature.

As mentioned above, these two denunciations are incompatible. Objects are
meaningless in the first but powerful enough to decide the fate of humans in
the second, while subjects are capable of shaping the natural world according
to their whims in the first and helplessly weak in the second. Social
scientists usually address this contradiction by positing a dualism in both
the objects and subjects: objects have a hard ``pole'' that is capable of
shaping and limiting behavior of the soft pole of subjects, and also a soft
pole upon which the hard pole of culture can project its values. Said
differently, subjects have a hard pole that is capable of projecting and
giving value to the soft pole of objects, and a soft pole that is shaped and
limited by the hard pole of the objects.

The problem with this is that the list of things that make up the hard and
soft parts are somewhat arbitrary. They depend on what the social science
believes. The hard parts of nature are whatever of the natural sciences they
believe, and the hard parts of society are whatever of the social sciences
they believe. 

When social scientists deployed the same arguments that were used to debunk
the soft components of nature on the hard components of nature (the sciences
themselves that were foundational aspects of their beliefs) by seeing them as
the products of the society's interests and requirements, the whole dualistic
paradigm collapsed. Once it was deemed absurd that society constructed all of
the scientific facts out of its own self-interest, it was also seen as much
less plausible that these soft facts could just as easily be dismissed in the
same manner.

Quasi-objects are offered as the solution to the problem that not all objects
and subjects are only either indefinitely malleable or infinitely powerful. To
take a direct quote from Latour,

``Quasi-objects are much more social, much more fabricated, much more
collective than the 'hard' parts of nature, but they are in no way the
arbitrary receptacles of a full-fledged society. On the other hand they are
much more real, nonhuman and objective than those shapeless screens on which
society – for unknown reasons – needed to be 'projected'.''

The proposition that will be expanded on below is that prosthetics and other
augmentations that enable PWDs to become more ``normal'' are quasi-objects.
Additionally that this has the potential to unify two different paradigms of
disability studies. 

\paragraph{Prosthetics and etc. are Quasi-objects}

The connection between a stump leg and a quasi-object was previously made by
(Bertram 2018). The proposition relevant for this discussion is that all
prosthetics are quasi-objects. Seen through the lense of the subject/object
dichotomy, a prosthetic is clearly not a subject. It is true that a prosthetic
itself is a nonhuman, nonliving, inorganic object, but seen through the
hard/soft pole dichotomy that Latour explains, is a prosthetic on the list of
hard or soft traits? 

In one sense, it belongs to the hard side because it is able to impact the
experience of subjects. Prosthetics allow for very real changes in the lives
of PWDs. Examples include enabling leg amputees to run, the deaf to hear, the
blind to see, etc. A prosthetic is not an intrinsically meaningless object
upon which society merely projects its values because it offers value to a PWD
independent of the society and culture. Thus the first denunciation is
debunked.

In another sense, it belongs to the soft side because it is the object of
human creation. Humans have designed this to suit their needs, and further
modifications can be made to reflect advancements in technology, ergonomics,
and user satisfaction. At any rate, a prosthetic cannot be said to be an all
powerful object over which humans have no control. Thus the second
denunciation mentioned above is debunked.

Once the two denunciations have been found to not apply, that this object is
neither exclusively on the hard pole nor the soft pole,  we find ourselves in
the territory of quasi-objects.


\paragraph{seeing prosthetics as a quasi-object can merge the social and
medical/functional model}

Conceding that a prosthetic limb is a quasi-object allows for the merging of
the social and medical/functional paradigms. 

\subparagraph{What are the premises of the paradigms?}

\subparagraph{How were they previously viewed as incommensurable? What were
the problems that the social model has with prosthetics?}

The social model seeks to rid PWDs of the ``sick person'' identity. In doing so
however the sick role is not fully removed, but transferred onto the broader
society. The problem doesn’t go away, it is simply reframed; a PWD is no
longer seen as sick and unable to care for him or herself, but rather society
at large is seen as being sick and unable to care for itself as a whole.

\subparagraph{How are they commensurable?}

\subparagraph{What practical considerations are involved too?}

cost of providing full prosthetics to everyone vs cost of building ramps
everywhere or some other more social solution

\subparagraph{Objections?}

A potential flaw in this line of argument may be that not all disabilities can
be removed with help from a prosthetic. However it is also the case that not
all disabilities can be removed via social interventions. I do concede that
prosthetics are not very helpful in the case of mental disabilities. 


\subsection{Datsonian perspective}

\subsubsection{Aperspectival objectivity}

\paragraph{``Objectively real'' - whatever survives the ``averaging of viewpoints
by communication,'' whereby individual idiosyncrasies are assumed to be
eliminated}

\paragraph{relate to the needs of the stakeholders, who is included in the
``averaging of viewpoints?''}





\begin{thebibliography}{AAA}
\bibitem{mypaper} \textsc{First, J.} and \textsc{Second, J.} (2010), ``An interesting paper'', \emph{A Famous Journal}, 1, pp.~1--11.
\bibitem{mybook} \textsc{First, J.} and \textsc{Second, J.} (2011), \emph{A great book}, Address: Publisher.
\end{thebibliography}
\end{document}
%%%%%%%%%%%%%%%%%%%%%%%%%%%%%%%%%%%%%%%%%%%%%%%%%%%%%%%%%%%%%%%%%%%%%%%%%%%%
\par

