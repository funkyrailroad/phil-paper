%%
%% This is file `lpstemplate.tex',
%% generated with the docstrip utility.
%%
%% The original source files were:
%%
%% lps.dtx  (with options: `template')
%%
%%  Copyright 2005, 2006 Gustavo Cevolani, Gruppo Utilizzatori Italiani di TeX
%%
%%  This program is provided under the terms of the
%%  LaTeX Project Public License distributed from CTAN
%%  archives in directory macros/latex/base/lppl.txt.
%%
%%  Author: Gustavo Cevolani
%%          at g.cevolani@gmail.com
%%
%%  This work has the LPPL maintenance status "author-maintained".
%%
%\documentclass{lps}
\documentclass[a4paper]{article}

\usepackage{lipsum}             % to test this template
%\usepackage[numbers,sort&compress]{natbib} %bib with automatic name and year
\usepackage[round]{natbib}      %bib with automatic name and year
\usepackage{hyperref}           %for creating links in the pdf version and other additional pdf attributes, no effect on the printed document
\usepackage[toc,page,title]{appendix}    % for appendix

\usepackage{indentfirst}             % indent first paragraph of section

\usepackage[utf8]{inputenc}     % support umlaut characters
\pagenumbering{roman}           % Start roman numbering

\begin{document}
%% Article's information:
\title{Inclusive Foreign Aid: Paradigm Commensuration, the Contextual
Distinction and Transdisciplinarity}
%% Author's information:
\author{Jake Atwell\\
    Freie Universität Berlin\\
    jake.atwell@gmail.com
}
\date{September 26, 2018}
%\shortauthor{John First, James Second}
%% Issue's information:
%%\Volume{}
%%\Number{}
%%\Year{}
%%\FirstPage{}
%%

\maketitle

%% Keywords and abstract:
%\keywords{some, crucial, terms.}


%\begin{abstract}
%Summary of the article.
%\end{abstract}
\newpage

%\noindent Research Question: What is inclusive foreign aid, is it a
%transdisciplinary field, and what insights are offered from a philosophy of
%science perspective?

\tableofcontents

\newpage
\pagenumbering{arabic} % Switch to normal numbers
%%

%%%%%%%%%%%%%%%%%%%%%%%% Main text: %%%%%%%%%%%%%%%%%%%%%%%%



\section{What Is Inclusive Foreign Aid?}

Inclusive foreign aid is foreign aid or development work that is aimed
primarily at advancing the status and welfare of people with disabilities
(PWDs) in low- and middle-income countries (LMICs). It is estimated that
approximately 15\% of the global population (1 billion people) has some form
of disability \citep{banks2017poverty}. The potential causes of a particular
disability are numerous. It can be a permanent condition since birth, the
result of an accident or a byproduct of the natural aging process. Lifelong
conditions can be blindness, deafness, muteness, lame limbs, etc. Disability
caused by an accident can be due to uncleared minefields, an amputation,
trauma, etc. Disabilities can also be temporary conditions, for example while
a person is recovering from an injury. The cause of a specific impairment does
not matter as much as the particular impairment because solutions address the
limitations imposed by the impairment as opposed to the cause of the
impairment. However, identifying causes is critical for identifying
appropriate preventative measures. PWDs face additional difficulties when
compared to their non-impaired brothers and sisters, and for this reason it is
conceivable that additional considerations are necessary if PWDs are to
benefit from typical aid projects.

\subsection{Recognized Problems}

Obstacles faced by PWDs in LMICs are numerous. A few of the prominent problems
addressed by the field of inclusive foreign aid are explained below.

\paragraph{Healthcare}

Immediately after an accident especially, but also in the case of standard
medical check-ups, PWDs need access to health facilities in order to be given
proper treatment. The world's poor in LMICs have limited access to health
facilities \citep{peters2008poverty}, and since those with disabilities are
disproportionately poor \citep{mitra2018disability}, this is a relevant
concern for inclusive foreign aid. Additionally, many health facilities are
not able to accommodate PWDs \citep{drainoni2006cross}. This could be due to
various physical barriers to entry such as no elevators or ramps. It could
also be a result of the hospitals simply not having the equipment to assist or
rehabilitate those with disabilities. For example, if a hospital does not have
any crutches, prosthetic legs or wheelchairs, then that hospital is not able
to help a person who lost their leg become more mobile.

\paragraph{Education}

Another obstacle faced by the world’s poor and disabled is a lack of basic
education. The reasons can be many: no nearby schools, the schools have
barriers that prevent PWDs from accessing or benefiting from them or the PWD
has a mental disability and is unable to progress in the particular
educational environment \citep{ingstad2011disability}. Regardless of the
reason, the consequences of having poor or incomplete education still persist.
One such consequence is that PWDs have fewer employment opportunities.

\paragraph{Employment}

In general, PWDs have a more difficult time finding and maintaining
employment. This is likely influenced by the fact that PWDs are
disproportionately poor and uneducated \citep{lamichhane2014nexus}. At the
very least, having a disability limits the range of possible employment
opportunities. If a person becomes disabled, the combination of their
disability and their employment will decided whether or not the person will be
able to continue working. \cite{chowdhury2006economics} observe that a
shopkeeper and farmer who sustain an injury that leads to the same disability
will have immensely different prospects depending on the disability. In the
case of a missing leg, the shopkeeper will largely be able to continue his
daily work as it is primarily sedentary, whereas the farmer will suffer far
worse consequences and likely no longer be able to continue. If a PWD is also
the head of household and primary breadwinner, the inability to work could
have dire consequences for the rest of the family \citep{world2011world}.

\paragraph{Credit}

PWDs in LMICs also have limited access to opportunities for credit. This is
not only true for PWDs, but generally for impoverished individuals in LMICs,
however poverty and disability are closely related
\citep{palmer2011disability}. This aspect is causally related to the reduced
employment opportunities aspect because it limits any entrepreneurial activity
that might be undertaken. The lack of legitimate credit options sets the scene
for predatory credit sharks to take advantage of those in need
\citep{beisland2017exploring}. Additionally, merely the availability of access
to credit does not solve the problem of the loan shark; its availability must
be made known to those in need.

\newpage
\section{Disability Paradigms}

Inclusive foreign aid borrows its conceptualizations of disability from
disability studies. There are two main conceptualizations thereof which embody
different approaches to solving problems associated with disability. These two
models are explained, contrasted and finally a suggestion is offered as to how
they might be merged or ``commensurated.''

\subsection{Introducing the Medical and Social Models} 
\label{models}

The first conceptualization of disability is referred to as the medical model.
This view represents how the rest of society came to view disability
historically. It views PWDs in a way similar to a ``sick person'' in that they
are relieved of certain social obligations but also deprived of certain social
rights. The disability is viewed as a deficiency that may be potentially
remedied via medical intervention so that the person may be able to live as
close to a ``normal'' life as possible. Possible interventions can include
prosthetics, implants and other augmentations of functional utility.

A second conceptualization that came as a reaction to the medical model is the
social model. It offers a social constructionist view of the situation whereby
a person is rendered disabled only by societal conditions and norms.
Proponents of the social model view medical interventions negatively because
of the implication that PWDs are insufficient or subhuman without such
interventions. They contend that PWDs are actively oppressed by people without
disabilities (whether intentionally or not). In essence, people without
disabilities (i.e. certain government officials and other bureaucrats) are
denying PWDs certain basic social rights. This is seen as an intolerable
injustice, and proponents of the social model wish to change whatever dominant
beliefs and practices of the time enable this mistreatment. The social model
can be summarized by two basic premises: 1) an impairment is made into a
disability only by social conditions and 2) any policy should focus on giving
the most autonomy to the impaired person as possible
\citep{winter2003development}.


\subsection{Conflict Between the Two Models}

The medical model and social model cannot be said to exist harmoniously. The
latter was formulated as a direct response to the former with every intent to
replace it, and therefore it should not be a surprise that they differ in some
important way. The two models are indicative of two different world views.
\cite{kuhn1970structure} claims that the world views held by the scientists
(their ways of conceptualizing the environment in which they do science)
influence the knowledge that will be generated in a given scientific endeavor.
A particular world view serves as the foundation upon which scientific
knowledge is built and Kuhn's claim is that different conceptual foundations
can lead to different knowledge despite both using a scientifically sound
methodology. Solutions from the medical model perspective are oriented at
changing characteristics of PWDs and solutions from the social model
perspective are oriented at changing the environment in which PWDs find
themselves. Neither approach is a priori superior, they merely reflect
differences in values.

Additionally, there are new linguistic elements. A critical element of the
social model not found in the medical model is the distinction between an
impairment and a disability. An impairment is defined as 
%
\begin{quote}
    \ldots the condition of a person lacking part or all of limb, organ or
    mechanism of the body. \citep{winter2003development} 
\end{quote}
%
A ``mechanism of the body'' is to be understood as a sense like hearing or
sight, but can also include more advanced functions like the ability to read.
Disability is defined as 
%
\begin{quote}
    \ldots the disadvantage or restriction of activity caused by a
    contemporary social organization which \ldots excludes [people with
    physical impairments] from participation in the mainstream of social
    activities. \citep{winter2003development}
\end{quote}
%
In this sense, an impairment does not necessitate a disability; the disability
only arises as a result of exclusion.\footnote{\cite{winter2003development}
offers an example of a person confined to a wheelchair seeking access to the
upper floors in a building. She is impaired by the above definition, but the
state of her disability depends on the environment. If the building has only
stairs, she is disabled because she is excluded from accessing the upper
floors. However if there is a functioning elevator, she is no longer excluded
and thus not disabled. With this set of definitions it is also possible for a
person to have a disability without having an impairment. An example of this
is a left-handed person who has difficulty using tools that are built for use
by a right-handed person.}

The medical model does not make a distinction between impaired and disabled
and it does not consider societal factors. If a proponent of the medical model
speaks with a proponent of the social model about ``disability,'' this is a
point about which they will simply be speaking past each other. The former
will understand it as being synonymous with ``impairment'' and the latter will
understand it as being related to the social environment. Although the
respective definitions can be provided fairly easily, it will not be easy for
either side to convince the other that their definitions are the correct ones,
or even of which criteria should be used to evaluate the correctness of the
definitions. These differences may indicate that the two viewpoints are
incommensurable.

\paragraph{Engineered Incommensurability}

If these two views are incommensurable, it is not exactly of the same type of
incommensurability about which Kuhn speaks. In his examples, there are at
least two coexisting schools of thought that only realize their differences
once confronted with an anomaly of some sort. The differing interpretations of
the anomaly lead to discussions in which the opposing sides try to convince
the other(s) that their interpretation is most appropriate.

In the example of the disability paradigms, this process is a bit different.
First a problem is identified that needs to be addressed: the exclusion of
PWDs. This can be viewed as similar to the anomaly identification step. The
default paradigm is analyzed and broken down into components. Because of
objections to the implications of the default paradigm, a new paradigm is
formulated that acts as an ideological alternative to the former. It is then
hoped that solutions based on this ideological alternative will be able to
solve the initial problem i.e. address the anomaly. This process is
fundamentally different; the two paradigms were not simultaneously present
before the anomaly and found to be incommensurable after discussion, rather
the latter was designed to be incommensurable with the former to prompt a
revolution. This kind of incommensurability is perhaps best called artificial
or engineered incommensurability because it is used instrumentally to achieve
a goal.

The process of engineered incommensurability is more directed and intentional
than those outlined by Kuhn. The motivations and objectives of the social
model's proponents are much more clearly visible than those of any of the
parties involved in the types of situations Kuhn explains. The social model's
primary purpose is to achieve a goal of greater inclusion for PWDs, not to
uncover the nature of disability. This goal is noble and worth pursuing, but
in Aristotle's view
%
\begin{quote}
    \ldots that which is desirable on its own account and for the sake of
    knowing it is more of the nature of Wisdom than that which is desirable on
    account of its results \ldots
\end{quote}
%
If the proponents of the social model uncover something about the nature of
disability that is deemed harmful to the goal of achieving greater inclusion
for PWDs, the motivation to achieve greater inclusion may take priority over
the motivation to more deeply understand the nature of disability.

\subsection{Paradigm Commensuration}

One consequence of engineered incommensurability is that the engineering may
have been done inadequately; it may be the case that these two viewpoints are
not actually incommensurable and are indeed fundamentally compatible with each
other. Conflicting beliefs in opposing world views can be the sources of
incommensurability and by resolving apparent conflicts between beliefs, world
views can be aligned and paradigms commensurated. It will be argued that the
medical and social models can be merged by using the notion of quasi-objects.

The social model argues that currently available facilities have been produced
by able-bodied people who have neglected the needs of PWDs, and that it is
society's duty to accommodate PWDs. It intends to do this by entirely
reframing society's view of PWDs and ridding them of the ``sick person''
identity. Once society has done this and built itself to no longer exclude
these people, ideally disability will be a notion of the past, and only
impairments will exist. In short, society has a social obligation to PWDs that
is currently not being met. This reframing of the problem however does not
solve the problem. The sick role is not fully removed from the situation, but
transferred onto the broader society; PWDs are no longer seen as sick and
unable to care for themselves, but rather society at large is seen as being
sick and unable to care for its members.

Additionally, important elements of the medical model are permitted in the
social model even though they conflict with basic tenants of the paradigm. In
the example of the person confined to a wheelchair used to explain the
difference between an impairment and a disability, it was explained that
disability exists only if the person is excluded. However, the element of the
medical model that is permitted by the social model is the wheelchair itself.
The wheelchair is a medical intervention that directly affects the
capabilities of the PWD independent of the social environment. In the case of
a person with no use of their limbs, this wheelchair is essential to their
ability to navigate a building at all; the elevator is not sufficient to
remove the disability in this case. If proponents of the social model wish to
be consistent in their aim that society should be made such that impairments
are not converted into disabilities, they make accomplishing that aim much
more difficult by condemning viable functional solutions for impairments.

The shortcomings of both the medical and social models can be addressed by
realizing that they are in fact complimentary. The medical model does not
consider societal factors and the social model restricts its applicability by
condemning medical interventions. By merging the two paradigms, it can stand
to widen its range of applicability. Conceding that physical medical
interventions are  quasi-objects allows for the merging of the social and
medical paradigms.\footnote{See the appendix for a refresher on quasi-objects
and why medical interventions fall into that category.} If one instead views
prosthetics and augmentations not merely as an inanimate objects, but rather
as a quasi-objects capable of fulfilling social roles and duties, proponents
of the social model can come to view some medical interventions as of a partly
social nature, and thus permissible in their paradigm.

\newpage
\section{The Contextual Distinction}

This field is currently an active field, and accordingly there are various
research efforts underway generating knowledge. This then begs the question,
how is knowledge generated in this field? One model of knowledge generation
involves a dichotomy of contexts: the context of discovery and the context of
justification. The context of discovery is associated with the ``Eureka!''
moment in which one ``discovers'' a new idea. The context of justification on
the other hand involves providing support for a given hypothesis or theory.
Traditionally the latter was seen as belonging more clearly to the domain of
philosophy of science; it relates to the epistemological concern of ``How do
we know what we know?'' \citep{schickore2014scientific}. Eventually the
distinction was challenged and there was further reflection on exactly what
philosophers meant when they spoke of this dichotomy of contexts
\citep{hoyningen2006context}. In the following, examples from inclusive
foreign aid of the traditional view of the contextual distinction will be
provided, as well as of how the two contexts are linked both to each other and
the theories that they helped create.

\subsection{Context of Discovery}

The context of discovery is the context in which ideas and knowledge claims
are generated. As inclusive foreign aid clearly has much in common with
foreign aid, one obvious way to generate knowledge claims is to borrow them
from the field of foreign aid. Although it has been found that many proposed
policies from foreign aid do not benefit PWDs as much as the rest of the
population (and thus the reason for the existence of \emph{inclusive} foreign
aid), these claims at least serve as valid starting points for useful
policies. Although the policies in their unchanged form may benefit PWDs less,
they may still benefit them, and there may be measures that can be taken to
increase those benefits. First consider the characteristics of a knowledge
claim relevant to this field.


\paragraph{A Knowledge Claim}

A knowledge claim is made in response to some unknown, the original question
that prompted the research. So one may begin by asking, what would the field
of inclusive foreign aid like to know? Or less abstractly, what would the
researchers and other relevant stakeholders in inclusive foreign aid like to
know? They want to know which policies are effective in a manner that allows
PWDs to also benefit. Simply put, they want to do good for PWDs (however
``good'' may be defined). Additionally, they want to do good for PWDs
\emph{efficiently}. There is a finite amount of money allocated for foreign
aid, and ideally it would be used in a nondiscriminatory manner such that the
largest number of people can benefit from it. As there is no criteria for the
absolute efficiency above which a certain policy should be deemed successful,
policies are usually judged with respect to other policies; the relative
efficiency is of interest. Given two policies to choose from, there needs to
be some way of choosing between the two. If policy A is able to provide a
benefit to 500 people for the same cost that policy B is able to provide the
same benefit to 5000 people, B is preferable to A. Thus we have established
two values of the inclusive foreign aid paradigm, doing good for PWDs in LMICs
and doing so in a cost-efficient manner. These values serve as criteria
against which any knowledge claims may be evaluated. The context of
justification is where this evaluation takes place, on which will be
elaborated shortly. Before this, a concrete example of a knowledge claim will
be provided.

Providing microfinance opportunities is a solution that was first implemented
in the field of foreign aid but can also potentially benefit populations of
people with disabilities. The idea behind microfinancing is that providing
very small loans at very low interest rates can provide struggling individuals
with large benefits \citep{wendt2006building}. Typical in the case of an
impoverished population is a self-perpetuating cycle of poverty. As an
example, a poor artisan may need to borrow to make a good which she can later
sell. However, she borrows the supplies and raw materials at a price that can
leave her with little or no profit, or even in debt. By taking advantage of
microfinancing opportunities, the artisan can increase her profit margins and
savings to a point where she no longer needs loans to sustain
herself.\footnote{Microfinancing opportunities are somewhat of a controversial
    development tool, but \cite{banerjee2015six} find them to have at least
``modestly positive, but not transformative, effects.''}

PWDs face additional obstacles to participation in such opportunities.
\cite{mersland2008access} outline five exclusion mechanisms by which PWDs are
afforded limited access to microfinancing opportunities: low self-esteem,
other members, staff, design, and physical and informational barriers. The
knowledge claim is that PWDs will be able to benefit more from microfinance
opportunities if these exclusion mechanisms are reduced or eliminated.

\subsection{Context of Justification}

The context of justification is the context in which the validity of a
particular knowledge claim is verified. In the sense of inclusive foreign aid,
the justification process amounts to determining whether or not a given policy
has the intended effects. Generally the policy goals are in line with reducing
poverty and suffering amongst those with disabilities, but the means by which
this is done and the intermediate goals can vary.

Absolutely crucial in any policy analysis effort is the availability and
analysis of useful metrics that at least serve as proxies for the states of
interest. For example, income may serve as a proxy for poverty. It is however
only a proxy because poverty is a multidimensional phenomenon that involves
variables other than merely income \citep{alkire2011understandings}. There are
two ways that this type of data may be obtained and interpreted: the micro and
macro perspectives \citep{ingstad2011disability}. The micro perspective
focuses on specific perspectives and stories, the concrete incidences and
consequences of living with a disability. The macro perspective considers
studies and surveys on key indicators for broader populations. Important for
both approaches are consistent definitions that can be used across studies to
ensure that apples are indeed being compared to apples.

\paragraph{A Justification Process}

The logic of a justification process is as follows. A good policy must
fulfill the two values of the inclusive foreign aid paradigm, so it must be
shown that it does good for PWDs and that it does so efficiently. In the case
of our specific knowledge claim above, it must be shown that lifting these
exclusion mechanisms leads to increased access for PWDs and that they benefit
from this increased access, as well as that this is an efficient means of
doing good for PWDs. In the following, addressing each exclusion mechanism
will be treated as a separate policy, and their success relative to each other
will determine the most efficient policy. 

A cost-benefit analysis (CBA) is a methodology that is capable of this sort of
analysis and comparison. CBA is a generalization of the typical accounting
work commonly done in many types of organizations. Traditional accounting
involves looking at expenditures and revenues to determine e.g. whether or not
a organization is profitable or if a project is on budget. CBA generalizes
this by not only considering the expenditures and revenues of a single
organization, but rather all parties involved. Additionally, it attempts to
incorporate elements that do not typically have dollar-valued costs associated
with them \citep{mishan2015elements} and simultaneously utilizes both macro
and micro perspectives of the situation to quantify every aspect that is
involved. In the end, the best policy is the policy that has the greatest
benefits for the least cost. Unfortunately, the only way to actually determine
the effectiveness of a policy is to implement it and observe the results.
These policy implementations are the experiments of inclusive foreign aid,
only after which can knowledge claims begin to be justified. 

Once the various policies have been implemented and the results measured can
the various costs and benefits for the relevant groups be considered and
compared. In the case of exclusion by physical and informational barriers, the
various costs considered could include the construction of ramps and
elevators, directed advertising and awareness campaigns. The various benefits
could include increased opportunities for PWDs and more customers for the
microfinance institution. Indirect costs and benefits can also be considered.
These may include the potential benefits that a more economically involved
population of PWDs would have on the broader society since having PWDs
financially participate would stimulate the economy. Considering indirect
costs can involve uncertainty, making assumptions and extrapolating from
incomplete data. This detracts from the precision of the methodology so these
issues should not be ignored, but even in the presence of uncertainty, CBA can
still provide likely candidates for the policy that does the most good by the
most efficient means. In the case of the particular exclusion mechanisms
outlined above, \cite{mersland2008access} found addressing exclusion from
self-esteem and staff to have the largest positive benefits for PWDs. However
their analysis lacks a quantitative comparison of the costs involved in
addressing each exclusion mechanism, so this is an area that needs to be
investigated further.

\paragraph{Practical Concerns}

Some comments on the practical difficulties involved in such a justification
process are warranted. For example, something as fundamental as the definition
of disability presents difficulties. Often seen as a binary description
(either disabled or not), there are clearly different types of disability.
Even amongst similar types, there are varying degrees of disability. The
challenge is defining disability in a manner than many can agree on while
neglecting none of these aspects.

Aside from the semantic problems involved with defining disability, there are
also difficulties involved with actually collecting this data. Inclusive
foreign aid suffers from a measurement problem and therefore has difficulties
in justifying whether or not given policies work. As
\cite{ingstad2011disability} note,
%
\begin{quote}
    \ldots a mapping and monitoring system is simply not in place in
    low-income countries, yielding a weak knowledge base for including people
    with disabilities as well as other vulnerable groups in research that can
    effectively inform poverty alleviation programmes.
\end{quote}
%
This practical obstacle is an important limitation for the context of
justification; it shows how young of a field inclusive foreign aid is and how
tentative any of its knowledge claims are.

\subsection{Inextricable Linkage}

Initially the contextual distinction was conceived as being one of a temporal
or methodological nature: the context of the discovery happens first and the
context of justification follows; one set of methods are used for discovery
and another for justification. This however is too simplistic.
\cite{kuhn1970structure} later showed that these contexts are not necessarily
distinct from another, that they can have common elements, and that they are
linked to the knowledge claims that they help generate. 

\paragraph{Non-exclusivity of the Contexts}
\label{part}

An example of a method that may be applied in both the discovery and
justification of a knowledge claim is known as participatory research.
Participatory research is a way of gathering information from a target
population in a way that enables the target population to contribute to and
shape the knowledge gleaned from the interaction
\citep{bergold2012participatory}. As an example, two anthropologists observing
an indigenous tribe can sit quietly and observe many situations without
interacting with any members of the tribe. This approach is limited in that it
only allows for knowledge to be generated from the perspective of the
anthropologists, and the perspective of the indigenous tribe is left
unutilized; the members of the tribe are merely the objects of the
anthropologists' observation. Participatory research incorporates these
unutilized perspectives into the knowledge-generation process. In the above
example, the anthropologists make their observations but then would continue
to engage in a dialogue with the indigenous people. In this way, the biases
introduced by the anthropologists via their inductive generalizations are
subjected to the input and criticism of the natives.\footnote{This
    introduction of new viewpoints allows for the ``averaging of viewpoints''
    process outlined by \cite{datson1992objectivity}, which she explains as
    being a way to reach aperspectival objectivity. Objectivity has
    historically been associated with different meanings, and aperspectival
    objectivity is that which loses the individual idiosyncrasies involved in
    the knowledge-generation process, i.e. the biases and preconceived notions
    held by the participants. Aperspectival objectivity has also been called
    ``the view from nowhere,'' since the elimination of idiosyncrasies can be
    said to amount to the elimination of the individual perspective. However,
    perhaps a more fitting name and metaphor for aperspectival objectivity is
    omniperspectival objectivity and ``the view from everywhere.`` The
    averaging process does not involve eliminating old elements, but rather
    incorporating new elements (i.e. perspectives), and for this reason can be
    seen as an additive operation. The typical mathematical operation
associated with averaging is the arithmetic mean, which involves addition, not
subtraction.}

As stated above, participatory research can be used in both the discovery and
justification contexts. In the context of discovery, observing and hearing the
input of the target population can be a great way to begin assessing the
situation of a given population. Without a firm grasp of what might even
potentially constitute a good solution to a given population's problem,
participatory research can be purely exploratory, void of the goal of
justifying a particular knowledge claim. On the other hand, participatory
research can function as a valid approach in the context of justification.
Previously, the two perspectives of analysis for assessing data for CBAs were
outlined: the micro and macro perspectives. The macro perspective approach
includes metrics like poverty and income levels. However, what should one do
if these data are not available? Similarly, how does one go about obtaining
such data? One must begin with the micro perspective. In the case of
addressing the exclusion mechanisms, participatory research offers an
extremely valuable perspective on whether or not a policy is doing good for
PWDs; it asks the PWDs directly. This is a much more direct approach than
going through the microfinance institution to see how a particular customer's
balance is changing and trying to interpret that in terms of the PWDs
wellbeing. While serving a justificatory purpose, if the proposed measure
fails to do good for the PWD, the aim of participatory research can then (in
the same conversation) turn into a more discovery oriented endeavor. It can
begin to explore possible explanations for why the proposed measure failed and
give rise to new ideas of possible exclusion mechanisms that may need to be
addressed. So not only can participatory research be employed in both contexts
of discovery and justification, it can be employed in both contexts
simultaneously.


\paragraph{Link to Knowledge Claim}

\cite{kuhn1970structure} argued that the two contexts are not fundamental
aspects of the knowledge generation process. In his view, the discovery and
justification processes are not elementary components of logic and
epistemology, but rather essential parts of the theories for which they played
an important role in forming. Kuhn explains how the concept of discovery is
actually a drawn out process and intertwined with theory. He explains this
based on an example of oxygen. There were multiple people who could be
considered candidates for the discovery of oxygen since they isolated it to
some extent. They may have known that they had produced and isolated something
new, but they did not necessarily know it was oxygen. Discovery and theory
become intertwined when one realizes that it is not necessarily the
``discoverer'' who proclaims himself as the discoverer in the instant of
discovery.\footnote{Kuhn also argued that first discovering something is
different than seeing something because it doesn't happen in an instant.
Discovery involves both realizing that one has discovered something, and
realizing what one has discovered. This can take time.} On the contrary, it
can be other people at a later point in time who come to a consensus about who
deserves the title of discoverer. Discoveries can be disruptive to the current
theory, and the theory may need to be adjusted to incorporate this new
discovery. Accordingly, the discoverer and the discovery are named and
interpreted within the view of the current state of the theory. A similar
situation arises in the justification process. In justifying the theory, one
uses elements of the theory to justify it.

In inclusive foreign aid, one can also see how elements of the theory are
integral parts of the justification process by examining how the notion of
poverty is used. Poverty has previously been defined simply as having income
below a certain level and a given knowledge claim was then justified with
respect to this simplistic definition. As the notion of poverty evolved and
its multi-dimensional nature came to be appreciated, certain knowledge claims
that had been justified with respect to the simplistic definition were no
longer justified with respect to the multi-dimensional definition. Nothing in
the logic of the justification process changed, but rather something prior to
the logic: one of the premises and definitions upon which the logical
argumentation was based. The context of justification surely involves logical
processes, but it also involves certain givens, and it is these givens that
are both part of the theory and part of the justification process.


\newpage

\section{Transdisciplinarity in Inclusive Foreign Aid}

Despite the common occurrence of words like multidisciplinarity,
interdisciplinarity and transdisciplinarity, they all have precise meanings.
The topics are explained and their meanings illustrated by an example, and
then it is shown how inclusive foreign aid is a transdisciplinary field.

\subsection{An Overview of Interdisciplinarity}

The concept of transdisciplinarity is a subtopic in the field of
interdisciplinarity. To get an idea of exactly what transdisciplinarity is, it
is helpful to have an overview of some basic concepts in the field of
interdisciplinarity. \cite{klein2010taxonomy} provides a detailed overview of
the different types of interdisciplinarity, from which the following borrows
heavily.

\paragraph{Disciplinarity}

Interdisciplinarity stands in contrast to the standard disciplines commonly
found in the modern university. These include classic subjects like physics,
chemistry, economics, psychology, etc. Although seemingly set in stone in the
modern university, these disciplines evolved into their current form over the
course of the last 200 years \citep{weingart2010short}. As is usual with
historical processes, their evolution is not entirely deterministic and if it
were to happen again, there stands the chance that the disciplines would have
emerged in a different form. Given the arbitrariness involved, one can wonder
which other disciplines could have emerged given certain other circumstances.
One can envision a combination, merging or even fracturing of the various
modern disciplines that might be more directed at solving slightly or
potentially even substantially different problems. Interdisciplinarity in a
broad sense represents the different ways this might occur.

\paragraph{Multidisciplinarity}

The most primitive form of interdisciplinarity is known as
multidisciplinarity. It can be thought of as the first step on the way to
transdisciplinarity from traditional academic disciplinarity.
Multidisciplinarity involves the perspective of multiple disciplines. By
incorporating multiple disciplines, the multidisciplinary approach provides
access to more knowledge and methods than a traditional disciplinary approach,
but each discipline remains fully intact and separate. In other words, there
is a lack of integration of the different perspectives. Words like
juxtaposing, sequencing and coordinating fit to the multidisciplinary
approach.

\paragraph{Interdisciplinarity}

A more advanced form of interdisciplinarity is known as interdisciplinarity.
The primary difference is the degree of integration of the individual
disciplinary perspectives that is achieved. An interdisciplinary approach can
involve borrowing the methods of one discipline and applying them to a problem
in a different discipline, however the degree to which this impacts the
borrowing discipline can vary. A stronger or more ``genuine'' form of
interdisciplinarity can be the result of theoretical integration. In this
sense of interdisciplinarity, one overarching theoretical framework can offer
insight into multiple disciplines, and findings in one discipline can
contribute to the problems and theories in another. Words like interacting,
integrating, focusing, blending and linking aptly describe this approach.

\paragraph{Transdisciplinarity}

Finally we arrive at transdisciplinarity. Transdisciplinarity, in the sense
relevant to this work, focuses on the generation of knowledge directed at
solving real-world problems. This amounts to letting the real-world and its
inhabitants dictate which problems are important and in need of solving as
opposed to the traditional academic disciplinary structures. Because of the
potential divergence between the needs of a specific real-world problem and
the specialization of a traditional discipline, it is possible that the
knowledge and methods needed to address a given real-world problem are spread
across different disciplines. This is a similar concept to
interdisciplinarity as explained above, with the additional condition that
external (to academia) stakeholders also contribute to the knowledge
generation process. As a result, transdisciplinarity can be thought of as
interdisciplinarity with societal input. Words like transcending,
transgressing and transforming describe this approach

\paragraph{An Instructive Example}

\cite{latour1999circulating} details an expedition in which a botanist,
pedologist and geomorphologist investigate the boundary between a forest and
savannah. They wish to determine the dynamics at the interface; is the forest
or savannah advancing, or perhaps neither? Upon first glance, the botanist's
analysis leads to the conclusion that forest could either be advancing or
receding, and the pedologist's analysis leads to the conclusion that the
savannah is advancing. Merely comparing the two conclusions from the two
disciplines amounts to a multidisciplinary approach. However, their
contrasting conclusions motivate further investigation. While the botanist and
pedologist are taking measurements and discussing the significance of the
results, the geomorphologist ``adds her two cents to all the conversations,
allowing her expatriate colleagues to 'triangulate' their judgments through
hers'' \citep{latour1999circulating}. This exchange is exemplary of an
interdisciplinary approach. There are three disciplines present, and each is
interacting with and integrating the viewpoints of the others. It differs from
the multidisciplinary approach in that there is an integrative exchange
between the individual disciplinary standpoints. In order to see
transdisciplinarity in action, there would have to be input from external
stakeholders. Latour mentions the significance that findings of this
investigation might have for potential investors, but this is not enough. In
order to display transdisciplinarity, the investors need to play a role in the
knowledge generation process.

\subsection{Inclusive Foreign Aid as a Transdisciplinary Field}

Inclusive foreign aid is a transdisciplinary field because it addresses a
societally relevant problem with the input of stakeholders and integrates
findings and methodologies from various disciplines. The genesis of this
problem clearly does not lie in any of the traditional disciplines. The entire
concept of foreign aid is focused on solving real-world problems, and as a
result there are stakeholders involved. The stakeholders in any endeavor
relevant to inclusive foreign aid are those funding, planning, implementing
and benefiting from any proposed measures. These categories are not
necessarily mutually exclusive. Ensuring that those receiving the aid play a
role in the knowledge generation process is in line with the slogan ``Nothing
about us without us'' found in the disability studies literature
\citep{pfeiffer2000disability}. The means by which the recipients play a role
in the knowledge generation process is known as participatory research, which
was explained in \autoref{part}.


Inclusive foreign aid incorporates aspects from a variety of different
disciplines including bionics, disability studies, economics, education,
engineering finance, law, medicine, psychology, sociology and technology. As
explained above, there is a difference between calling upon the knowledge and
methodologies of those disciplines in a merely juxtapositional manner and
doing so in a more integrative manner (i.e. multidisciplinarity vs.
interdisciplinarity). Inclusive foreign aid borrows many concepts and methods
from disability studies. Some of these things include the paradigms and models
of disability and the survey methods. The issue of how to effectively conduct
surveys is a good example of borrowing a methodology from one domain and
adapting it to fit the current domain. Inclusive foreign aid also borrows
concepts in economics of poverty, inequality, asset index\footnote{An asset
    index measures wealth not merely by income, but also considers basic
assets like access to basic sanitation, running water, indoor plumbing, etc.}
and economic mobility. Viewing disability through the lens of inequality
allows one to determine its influence on economic welfare. For example, by
measuring the asset index in different countries and across income groups,
PWDs were found to be two to four times more likely to be in the lowest income
quintile than in the highest income quintile \citep{mitra2018disability}. As a
final example, inclusive foreign aid borrows concepts from the field of
technology in the way of prosthetics and other assistive technologies
\citep{roulstone2016disability}. These technologies, in very simple cases,
have the potential eliminate a disability. In surveys of Ethiopia, Malawi,
Tanzania and Uganda, the most common disability for those with both severe and
moderate functional difficulties was sight-related
\citep{mitra2018disability}. It was also found that assistive technologies
were largely unused due to either cost of availability issues. Simple measures
such as glasses would be sufficient to greatly reduce the number of people who
identify themselves as disabled.



%%%%%%%%%%%%%%%%%%%%%%
%%%%%%%%%%%%%%%%%%%%%%
%%
%% Appendix
%%
%%%%%%%%%%%%%%%%%%%%%%
%%%%%%%%%%%%%%%%%%%%%%

\newpage
\begin{appendices}
\section{Quasi-objects}

\cite{latour2012we} proposes the idea of a quasi-object as a solution to a
problem with the traditional subject/society and object/nature dichotomy.
There are two faulty beliefs refuted by contradictory denunciations when it
comes to the interaction between objects and subjects.
%
\begin{itemize} \item[] Belief 1: Objects have an intrinsic value (i.e. money,
            gods, status, etc).

    \item[] Denunciation 1: Those objects are intrinsically meaningless and
        serve merely as blank slates onto which society projects its values.
\end{itemize}
%
In this view of things, the objects are weak and worthless, and society is
powerful enough to be capable of making anything out of them. 
%
\begin{itemize}
    \item[] Belief 2: Subjects have free will and the power to do whatever it
        is they like.

    \item[] Denunciation 2: There are natural constraints on subjects that
        place limitations on what they can do. 
\end{itemize}
%
In this view of things, subjects are weak over the influence of an
overwhelmingly powerful nature. The two denunciations are incompatible because
objects are meaningless in the first but powerful enough to influence the fate
of humans in the second, while subjects are capable of shaping the natural
world to an arbitrary degree in the first and powerless in the second.

This contradiction is traditionally addressed by positing a dualism in both
the objects and subjects: objects have a hard ``pole'' that is capable of
shaping and limiting behavior of the soft pole of subjects, and also a soft
pole upon which the hard pole of culture can project its values. Said slightly
differently, subjects have a hard pole that is capable of projecting and
giving value to the soft pole of objects, and a soft pole that is shaped and
limited by the hard pole of the objects. The problem with this is that the
list of things that make up the hard and soft parts are somewhat arbitrary.
The hard parts of nature are whatever of the natural sciences one believes,
and the hard parts of society are whatever of the social sciences one
believes.

Quasi-objects are offered as the solution to the problem that not all objects
and subjects are only either indefinitely malleable or infinitely powerful
i.e. ``soft'' or ``hard.'' To take a direct quote from Latour,
%
\begin{quote}
Quasi-objects are much more social, much more fabricated, much more
collective than the 'hard' parts of nature, but they are in no way the
arbitrary receptacles of a full-fledged society. On the other hand they are
much more real, nonhuman and objective than those shapeless screens on which
society – for unknown reasons – needed to be 'projected.' \citep{latour2012we}
\end{quote}
%
In essence, quasi-objects are objects that also fulfill some social role.


\newpage
\section{Prosthetics as Quasi-objects}
\label{pros-quasi}

The connection between a stump leg and a quasi-object was previously made by
\cite{bertram2018bestial}. The proposition is that physical medical
interventions (i.e. wheelchairs, prosthetics, implants and other
augmentations) are quasi-objects. Seen through the lens of the subject/object
dichotomy, they are clearly not a subjects because they are inanimate objects.
It is true that they are nonhuman, nonliving, inorganic objects, but seen
through the hard/soft pole dichotomy that Latour explains, are they on the
list of hard or soft traits?

In one sense, they belong to the hard side because they are able to impact the
experience of subjects. They allow for very real changes in the lives of PWDs.
Examples include enabling leg amputees to run, the deaf to hear, the blind to
see, etc. They are not intrinsically meaningless objects upon which society
merely projects its values because they offer value to a PWD independent of
the society and culture. Thus the first denunciation is debunked. In another
sense, they belong to the soft side because they are the objects of human
creation. Humans have designed them to suit their needs, and further
modifications can be made to reflect advancements in technology, ergonomics,
and user satisfaction. At any rate, they cannot be said to be all powerful
objects over which humans have no control. Thus the second denunciation above
is debunked. Once the two denunciations have been found to not apply, i.e.
that they are neither exclusively on the hard pole nor the soft pole, we find
ourselves in the territory of quasi-objects.

\end{appendices}






\cleardoublepage
\bibliography{mybib}
%\bibliographystyle{plainnat}
\bibliographystyle{apalike}
%\bibliographystyle{ieeetr}
%\bibliographystyle{detailed}

%\begin{thebibliography}{AAA}
%\bibitem{mypaper} \textsc{First, J.} and \textsc{Second, J.} (2010), ``An interesting paper'', \emph{A Famous Journal}, 1, pp.~1--11.
%\bibitem{mybook} \textsc{First, J.} and \textsc{Second, J.} (2011), \emph{A great book}, Address: Publisher.
%\end{thebibliography}

\end{document}
%%%%%%%%%%%%%%%%%%%%%%%%%%%%%%%%%%%%%%%%%%%%%%%%%%%%%%%%%%%%%%%%%%%%%%%%%%%%
\par

