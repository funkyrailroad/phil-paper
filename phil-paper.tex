%%
%% This is file `lpstemplate.tex',
%% generated with the docstrip utility.
%%
%% The original source files were:
%%
%% lps.dtx  (with options: `template')
%%
%%  Copyright 2005, 2006 Gustavo Cevolani, Gruppo Utilizzatori Italiani di TeX
%%
%%  This program is provided under the terms of the
%%  LaTeX Project Public License distributed from CTAN
%%  archives in directory macros/latex/base/lppl.txt.
%%
%%  Author: Gustavo Cevolani
%%          at g.cevolani@gmail.com
%%
%%  This work has the LPPL maintenance status "author-maintained".
%%
%\documentclass{lps}
\documentclass[a4paper]{article}

\usepackage{lipsum}             % to test this template
%\usepackage[numbers,sort&compress]{natbib} %bib with automatic name and year
\usepackage[round]{natbib}      %bib with automatic name and year
\usepackage{hyperref}           %for creating links in the pdf version and other additional pdf attributes, no effect on the printed document
\usepackage[title]{appendix}    % for appendix


\usepackage[utf8]{inputenc}     % support umlaut characters
\pagenumbering{roman}           % Start roman numbering

\begin{document}
%% Article's information:
\title{Knowledge generation and transdisciplinarity in inclusive foreign aid
and perspectives from the philosophy of science}
%% Author's information:
\author{Jake Atwell\\
    Freie Universität Berlin\\
    jake.atwell@gmail.com
}
\date{September 26, 2018}
%\shortauthor{John First, James Second}
%% Issue's information:
%%\Volume{}
%%\Number{}
%%\Year{}
%%\FirstPage{}
%%

\maketitle

%% Keywords and abstract:
%\keywords{some, crucial, terms.}


\begin{abstract}
Summary of the article.
\end{abstract}
\newpage

\noindent Research Question: What is inclusive foreign aid, is it a
transdisciplinary field, and what insights are offered from a philosophy of
science perspective?

\tableofcontents

\newpage
\pagenumbering{arabic} % Switch to normal numbers
%%

%%%%%%%%%%%%%%%%%%%%%%%% Main text: %%%%%%%%%%%%%%%%%%%%%%%%



\section{What is inclusive foreign aid?}

Inclusive foreign aid is foreign aid or development work that is aimed
primarily at advancing the status and welfare of people with disabilities
(PWDs) in low- and middle-income countries (LMICs). It is estimated that
approximately 15\% of the global population (approximately 1 billion people)
has some sort of disability \citep{banks2017poverty}.

The potential causes of a particular disability are numerous. It can be a
permanent condition since birth, the result of an accident or a byproduct of
the natural aging process. Lifelong conditions can be blindness, deafness,
muteness, lame limbs, etc. Disability caused by an accident can be due to
uncleared minefields, an amputation, trauma, etc. Disabilities can also be
temporary conditions for example as a result of an injury while the person is
recovering. The cause of a specific impairment does not matter as much as the
particular impairment because solutions address the limitations imposed by the
impairment as opposed to the cause of the impairment. However, identifying
causes is critical for identifying appropriate preventative measures.

PWDs face additional difficulties when compared to their non-impaired brothers
and sisters, and for this reason it is conceivable that additional
considerations are necessary if PWDs are to benefit from typical development
aid projects.


\subsection{Recognized problems}

\paragraph{Healthcare}

Obstacles faced by PWDs in LMICs  are numerous. Immediately after an accident
especially, but also in the case of standard medical check-ups, PWDs need
access to health facilities in order to be given proper treatment if needed.
The world's poor in LMICs have limited access to health facilities
\citep{peters2008poverty}, and since those with disabilities are
disproportionately poor \citep{mitra2018disability}, this is a relevant
concern for inclusive foreign aid. Additionally, many health facilities are
not able to accommodate PWDs \citep{drainoni2006cross}. This could be due to
various barriers to entry, which could include a lack of elevator, only stairs
outside and no ramp. It could also be a result of the hospitals simply not
having the equipment to assist or rehabilitate those with disabilities. For
example, if a hospital does not have any crutches, prosthetic legs or
wheelchairs, then that hospital is not able to help a person who just lost
their leg become more mobile.

\paragraph{Education}

Another obstacle faced by the world’s poor and disabled is a lack of basic
education. The reasons can be myriad: no nearby facilities, the facilities
have barriers that prevent PWDs from accessing or benefiting from them, the
PWD is mentally deficient in some respect and unable to progress in an
educational environment suited to more normally-abled people, or because of
poverty \citep{ingstad2011disability}. Regardless of the reason, the consequences are the
same. One such consequence is that PWDs have fewer employment opportunities.

\paragraph{Employment}

In general, PWDs have a more difficult time finding and maintaining
employment. This is likely influenced by the fact that PWDs are
disproportionately poor and uneducated \cite{lamichhane2014nexus}. At the very
least, having a disability limits the range of possible employment
opportunities. If a person becomes disabled, the combination of their
disability and their employment will decided whether or not the person will be
able to keep their job. \cite{chowdhury2006economics} observe that a
shopkeeper and farmer who sustain an injury that leads to the same disability
will have immensely different prospects depending on the disability. In the
case of a missing leg, the shopkeeper will largely be able to continue his
daily work as it is primarily sedentary, whereas the farmer will suffer far
worse consequences and likely no longer be able to continue. If a PWD is also
the head of a household and the primary breadwinner, the inability to work
could have dire consequences for the whole family \citep{world2011world}.


\paragraph{Credit}

PWDs in LMICs also have limited access to opportunities for credit. This is
not only true for PWDs, but generally for impoverished individuals in LMICs,
however poverty and disability are closely related
\citep{palmer2011disability}. This aspect is causally related to the no/fewer
employment opportunities aspect because it limits any entrepreneurial activity
that might be undertaken. The lack of legitimate credit options sets the scene
for predatory credit sharks to take advantage of those in need
\citep{beisland2017exploring}. Additionally, merely the availability of access
to credit does not solve the problem of the loan shark; its availability must
be made known to those in need.


\newpage
\section{Disability paradigms}
\label{models}

\paragraph{Medical and social models}
Inclusive foreign aid is primarily concerned with PWDs, and as such borrows
its conceptualizations of disability from disability studies. There are a
number of paradigms within disability studies, and also a number of problems
that the field is oriented towards solving. The different paradigms represent
different approaches to solving these problems. 
ANOTHER SENTENCE HERE \cite{winter2003development} outlines the two models.

The first conceptualization of disability is referred to as the medical model.
There was not an active attempt to formulate this view, but it represents
historically how the rest of society came to view PWDs. It views a person with
a disability in a way similar to a sick person. The disability is viewed as a
deficiency that may be potentially remedied via medical intervention so that
the person may be able to live as close to a ``normal'' life as possible.

A second paradigm that came as a reaction to the medical model is called the
social model. It offers a social constructionist view of the situation whereby
the person with disability (in the sense of the medical model) is only
rendered disabled by the societal conditions and norms. The social model
offers a distinction between an impairment and a disability. An impairment can
be defined as 
%
\begin{quote}
    \ldots the condition of a person lacking part or all of limb, organ or mechanism
    of the body.
\end{quote}
%
A mechanism of the body is to be understood as a sense like hearing or sight,
but can also include something more advanced like the ability to read.
Disability is then defined as 
%
\begin{quote}
    \ldots the disadvantage or restriction of activity caused by a contemporary
    social organization which \ldots excludes [people with physical
    impairments] from participation in the mainstream of social activities.
\end{quote}
%
In this sense, an impairment does not necessitate a disability; the disability
arises only if they are somehow excluded. For example, a person confined to a
wheelchair is impaired by the above definition, but the state of their
disability is context-specific. If this person is in a building without an
elevator and needs to gain access to the roof, he or she is disabled. However,
the disability disappears if there is a functioning elevator. Conversely, with
this set of definitions, it is also possible for a person to have a disability
without being impaired. An example of this is a left-handed person who has
difficulty using tools that are traditionally built for use by a right-handed
person.

A consequence of this exclusion is that PWDs are being actively oppressed
(whether intentionally or not) by people without disabilities. In essence,
people without disabilities (i.e. certain government officials and other
beaurocrats) are denying PWDs certain basic social rights. This is seen as an
intolerable injustice, and the goal becomes changing whatever dominant beliefs
and practices of the time enable this. \cite{winter2003development} condenses
the social model down to two basic premises: 1) an impairment is made into a
disability only by social conditions and 2) any policy should focus on giving
the most autonomy to the impaired person as possible.

The social model and medical model cannot be said to exist harmoniously. The
social model was formulated as a direct response to the medical model with
every intent to demolish it, and therefore it should not be a surprise that
they differ in some fundamental way. This fundamental difference is the source
of the incommensurability of the two viewpoints. However, the
incommensurability of these two viewpoints is not exactly of the same type
about which Kuhn speaks. The Kuhnian incommensurability is the result of some
fundamental difference 
LEFT OFF HERE



As Kuhn realized, a scientific methodology is not sufficient to uniquely
determine the knowledge that will be generated in a given scientific endeavor.
The world views held by the scientists (their ways of conceptualizing the
environment in which they do science) serve as the foundation upon which their
scientific knowledge is built. Kuhn's claim is that different foundations can
lead to different structures despite using similar tools (scientifically valid
methods).



\subsection{Paradigm commensuration}

These foundational world views can be viewed as being composed of postulates.
Conflicting postulates in opposing world views are the source of the
incommensurability. By resolving conflicts between postulates, world views can
be aligned and paradigms merged. An attempt at merging the medical and social
models by addressing their fundamental assumptions will be outlined below.

The social model seeks to rid PWDs of the ``sick person'' identity. In doing
so however the sick role is not fully removed, but transferred onto the
broader society. The problem does not go away, it is simply reframed; a PWD is
no longer seen as sick and unable to care for him or herself, but rather
society at large is seen as being sick and unable to care for all of its
members.

The social model argues that it is society's duty to accommodate for people
with impairments, as the current incarnation of available facilities has been
generated by able-bodied people who neglected the needs of impaired persons.
In doing this, the social model hopes to entirely reframe society's view of a
PWI. Once society has done this and built itself to no longer exclude these
people, ideally disability will be a notion of the past, and only impairments
will exist. This is done on the initiative of the larger society, does not
require labeling an impaired person as a ``sick'' person.

However, we see that even in examples given by proponents of the social model,
elements of the medical model are still assumed to be relevant. In the
example of a person confined to a wheelchair described above used to explain
the difference between an impairment and a disability, a disability was
defined to exist only if an impaired person is somehow excluded as a result of
their impairment. To repeat, this person in a wheelchair is disabled in a
building in which there are only stairs, but not disabled if there are
accessible elevators. However, the element of the medical model that is
assumed by the social model is the wheelchair itself. In the case of a person
with no use of their limbs, this wheelchair is essential to their ability
to navigate a building at all; the elevator is not sufficient to remove the
disability.

Conceding that a prosthetic limb is a quasi-object allows for the merging of
the social and medical/functional paradigms.

The social model generally views prosthetics and augmentations unfavorably
because the implication is that the person with impairment is inadequate,
favoring the alternative view that society itself is inadequate. If one
instead views prosthetics and augmentations not merely as an inanimate
objects, but rather as a quasi-objects that also fulfill a social role and
duty, perhaps the proponents of the social model would be more satisfied with
solutions of that form.

\subparagraph{Objections?}

A potential flaw in this line of argument may be that not all disabilities can
be removed with help from a prosthetic. However it is also the case that not
all disabilities can be removed via social interventions. I do concede that
prosthetics are not very helpful in the case of mental disabilities.

\subparagraph{What practical considerations are involved too?}

Theoretically prosthetics are a tenable solution, but practical matters such as
cost and ease of implementation will also have to be considered.

cost of providing full prosthetics to everyone vs cost of building ramps
everywhere or some other more social solution


\newpage
\section{Knowledge generation and the contextual distinction}

This field is currently an active field, and accordingly there are various
research efforts underway generating knowledge. This then begs the question,
how is knowledge generated in this field?


One model of knowledge generation involves a dichotomy of contexts: the
context of discovery and the context of justification. The context of
discovery is associated with the ``Eureka!'' moment in which one ``discovers''
a new idea. The context of justification on the other hand involves providing
support for a given hypothesis or theory. Traditionally the latter was seen as
belonging more clearly to the domain of philosophy of science; it relates to
the epistemological concern of ``How do we know what we know?''
\citep{schickore2014scientific}. Eventually the distinction was challenged and
there was further reflection on exactly what philosophers meant when they
spoke of this dichotomy of contexts \citep{hoyningen2006context}. In the
following, examples of the traditional view of the contextual distinction will
be provided, as well as of how they are linked both to each other and the
theories that they helped create.

\subsection{Context of discovery}

The context of discovery is the context in which ideas and knowledge claims
are generated. As inclusive foreign aid clearly has much in common with
foreign aid, one obvious way to generate knowledge claims is to borrow them
from the field of foreign aid. Although it has been found that some of these
proposed measures do not benefit PWDs as much as the rest of the population
(and hence this entire field), these claims at least serve as valid starting
points for useful policies. Although the policies in their unchanged form may
benefit PWDs less, they may still benefit, and there may be measures that can
be taken to increase those benefits. First we'll take a look at the
characteristics of a knowledge claim relevant to this field.


\paragraph{The knowledge claim}

One must know the precise knowledge claim. The knowledge claim is made in
response to some unknown, the original question that prompted the research. So
one may begin by asking, what would the field of inclusive foreign aid like to
know? Or less abstractly, what would the researchers and other relevant
stakeholders in inclusive foreign aid like to know? They want to know which
policies are effective in a manner that allows PWDs to also benefit. Simply
put, they want to do good for PWDs (however ``good'' may be defined.)
Additionally, they want to do good for PWDs \emph{efficiently}. There is a
finite amount of money allocated for foreign aid, and ideally it would be used
in a nondiscriminatory manner such that the largest number of people can
benefit from it. There is no criteria for the absolute efficiency above which
a certain policy is deemed successful. Policies are usually judged with
respect to other policies; the relative efficiency is of interest. Given two
policies to choose from: policy A and policy B, there needs to be some way of
choosing between the two. If A is able to provide a service to 500 people for
the same cost that B is able to provide the same service to 5000 people, B is
preferable to A. Thus we have established two values of the inclusive foreign
aid paradigm, doing good for PWDs in LMICs and doing so in a cost efficient
manner. These values serve as criteria against which any knowledge claims may
be evaluated. The context of justification is where this evaluation takes
place, on which will be elaborated shortly. Before this, a concrete example of
a knowledge claim will be provided.

\paragraph{Microfinance opportunities}

% What do I want to say in this section? I want to introduce microfinance as a
% way to do good for poor people in LMICs. I want to cite sources saying that
% PWDs have been shown not to be able to benefit from these measures as much
% as able-bodied people. The knowledge claim is that by addressing these
% exclusion mechanisms, PWDs can benefit more from these microfinance
% opportunities (first value). The knowledge claim relevant to the second
% value would be to identify the exclusion mechanism that has the largest
% impact.

Providing microfinance opportunities is a solution that was first implemented
in the field of foreign aid but can also potentially benefit populations of
people with disabilities. The idea behind microfinancing is that providing
very small loans can enable struggling individuals to break out of this
cycle \citep{wendt2006building}. Typical in the case of an impoverished
population is a self-perpetuating cycle of poverty. As an example, a poor
person who does not have enough money will need to borrow to make a good which
they can later sell. However, he borrows the supplies and raw materials at a
price that can leave him with little or no profit, or even in debt.
Microfinancing opportunities are somewhat of a controversial development tool,
but \cite{banerjee2015six} find them to have at least ``modestly postive, but
not transformative, effects.''

However, PWDs face additional obstacles to participation in such
opportunities. \cite{mersland2008access} outline five exclusion mechanisms by
which PWDs are afforded limited access to microfinancing opportunities: low
self-esteem, other members, staff, design, and physical and informational
barriers. The knowledge claim is that PWDs will be able to benefit more from
microfinance opportunities if these exclusion mechanisms are reduced or
eliminated.

\subsection{Context of justification}

The context of justification is the context in which the validity of a
particular knowledge claim is verified.

In the sense of inclusive foreign aid, the justification process amounts to
determining whether or not a given policy has the intended effects. Generally
the policy goals are in line with reducing poverty and suffering amongst those
with disabilities, but the means by which this is done and the intermediate
goals can vary. Absolutely crucial in any policy analysis effort is the
availability and analysis of useful metrics that at least serve as proxies for
the states of interest. For example, income may serve as a proxy for poverty.
It is however only a proxy because poverty is a multidimensional phenomenon
that involves variables other than merely income
\citep{alkire2011understandings}.

There are two ways that this type of data may be obtained and interpreted: the
micro and macro perspectives \citep{ingstad2011disability}.
%Here is also the contrast between the pure abstractions of the ``universal''
%and the ``particular,'' as well as the hybrid ``individual'' as used by
%\cite{collingwood1922history}.
The micro perspective focuses on specific perspectives and stories; the
concrete incidences and consequences of living with a disability. The macro
perspective considers studies and surveys on key indicators for broader
populations. Important for both approaches are consistent definitions that can
be used across studies to ensure that apples are indeed being compared to
apples.

\paragraph{Practical Concerns}

Before one can outline a justification process for the field, some comments on
the practical difficulties involved are warranted. For example, something as
fundamental as the definition of disability presents difficulties.
Traditionally seen as a binary description, either disabled or not, there are
clearly different types of disability. Even amongst similar types, there are
varying degrees of disability. The challenge is defining disability in a
manner than many can agree on while neglecting none of these aspects.

Aside from the semantic problems involved with defining disability, there are
also practical problems involved with actually collecting this data.
Inclusive foreign aid suffers from a measurement problem and therefore has
difficulties in justifying whether or not given policies work. As
\cite{ingstad2011disability} note,
%
\begin{quote}
    \ldots a mapping and monitoring system is simply not in place in
    low-income countries, yielding a weak knowledge base for including people
    with disabilities as well as other vulnerable groups in research that can
    effectively inform poverty alleviation programmes.
\end{quote}
%
This practical obstacle is an important limitation for the context of
justification; it shows how young of a field inclusive foreign aid is and how
tentative any of its knowledge claims are.


What do I want to do in this section?
- Explain the context of justification
    - what it means to justify a knowledge claim
    - generally how a claim could be justified in this field
        - could have the macro and micro views here (dump collingwood
          reference)
        - difficulties involved (measuring problem)
- show how the example knowledge claim could be justified.
    - show that ``good'' has been done
        - in the study, only increasing proportion of enrollments of PWDs was
          seen, this is incomplete for above paradigm (macro view)
            - full macro view not feasible
                - incorporating all the variables becomes grueling for the
                participating agencies.
        - rely on testimonials (participatory research) (micro view)
    - show that this has been achieved efficiently
        - cost benefit analysis
-






\paragraph{Cost-Benefit Analysis}
\paragraph{A justification process}

To justify the claim that PWDs will benefit more from microfinance
opportunities if the exclusion mechanisms are addressed, it needs to be
evaluated with respect to the two values of the inclusive foreign aid
paradigm. It must do good for PWDs, and it must do so efficiently. A
cost-benefit analysis (CBA) is a methodology that is capable of evaluating
this.

CBA is a generalization of the typical accounting work commonly done in many
types of organizations. Traditional accounting involves looking at
expenditures and revenues to determine for example if an organization is
profitable or if a project is on budget. CBA generalizes this by not only
considering the expenditures and revenues of a single organization (i.e. group
of stakeholders), but rather all parties involved. Additionally, it attempts
to incorporate goods that do not typically have dollar-valued costs associated
with them \citep{mishan2015elements}. It simultaneously utilizes both macro-
and micro-level views of the situation to quantify every aspect that is
involved.

The logic of the justification process is as follows.

A good policy must fulfill the two values of the of the inclusive foreign aid
paradigm, so it must be shown that it does good for PWDS, and that it does so
efficiently. In the case of our specific knowledge claim above, it must be
shown that lifting these exclusion mechanisms leads to increased access for
PWDs and that they benefit from this increased access, as well as that this is
an efficient means of doing good for PWDs. In the following, addressing each
exclusion mechanisms will be treated as a separate policy, and their success
relative to each other will determine the most efficient policy.

Unfortunately, the only way to really gauge the effectiveness of a policy is
to implement it and observe the results. These policy implementations are the
experiments of inclusive foreign aid, only after which can knowledge claims
begin to be justified. Once the various policies have been implemented and
the results measured, we may begin tallying the various costs and benefits for
the relevant groups. In the case of exclusion by physical and informational
barriers, the various costs considered could include the construction of ramps
and elevators, directed advertising and awareness campaigns. The various
benefits could include increased opportunities for PWDs and more customers for
the MFI. Indirect costs and benefits can also be considered. These could
include the potential benefits that a more economically involved population of
PWDs would have on the broader society. Providing PWDs with an opportunity to
financially participate would stimulate the economy. Considering indirect
costs can involve uncertainty, making assumptions and extrapolating from
incomplete data. This detracts from the precision of the methodology so these
issues should not be ignored, but even in the presence of uncertainty, CBA can
still provide likely candidates for the policy that does the most good by the
most efficient means.





STEPS
1. Identify if exclusion mechanisms are really excluding people
    - ask people (self perception)
        - flaw: people are not always the best judge of their own situation
            - some people might attribute their unemployment/firing to their
            disability when in fact it could be for other reasons
    - see if there is increased access once relevant barriers are
      lifted/reduced
        - have to implement the policy measures and then measure the
          difference
        - assumedly it would be wise to do a CBA of the options to see which
        are the most efficient to test and address, eliminate the immensely
        costly options from the beginning if there is no chance that they also
        have immensely large benefits

2. Judge which is the most efficient option to address



\subsection{Inextricable linkage}


Initially the context distinction was conceived as being one of a temporal or
methodological nature. The context of the discovery happens first and then the
context of justification follows; one set of methods are used for discovery
and another set are used for justification. This however is too simplistic.
It was argued that these contexts are not necessarily distinct from another,
that they can have common elements, and that they become linked to the
knowledge claim that they helped generate. An example of a method that may be
applied in both the discovery and justification of a knowledge claim is known
as participatory research.

\paragraph{Participatory research}
\label{part}

Participatory research is a way of gathering information from a target
population in a way that enables the target population to contribute to and
shape the knowledge gleaned from the interaction
\cite{bergold2012participatory}. As an example, two anthropologists observing
an indigenous tribe can sit quietly and observe many situations without
interacting with any members of the tribe. This approach is limited in that it
only allows for knowledge to be generated from the perspective of the
anthropologists, and the perspective of the indigenous tribe is left
unutilized; the members of the tribe are merely the objects of the
anthropologists' observation.

Participatory research intends to incorporate these unutilized perspectives
into the knowledge-generation process. In the above example, the
anthropologists make their observations but then would continue to engage in a
dialogue with the indigenous people. In this way, the biases introduced by the
anthropologists via their inductive generalizations are subjected to the input
and criticism of the natives. This introduction of a new viewpoint allows for
the ``averaging of viewpoints'' process that is outlined by
\cite{datson1992objectivity}. According to Datson, this process is the road to
aperspectival objectivity. Objectivity has historically been associated with
different meanings, and aperspectival objectivity is that which loses the
individual idiosyncrasies involved in the knowledge-generation process, i.e.
the biases and preconceived notions held by the observers. Aperspectival
objectivity has also been disparagingly dubbed ``the view from nowhere,''
since the elimination of idiosyncrasies can be said to amount to the
elimination of the individual perspective. However, this author finds a more
fitting metaphor for aperspectival objectivity to be ``the view from
everywhere.`` The averaging process does not involve eliminating old elements,
but rather incorporating new elements (i.e. perspectives), and for this
reason should be seen as an additive operation. The typical mathematical
operation associated with averaging is the arithmetic mean. In this process, N
different values, which can represent anything, are all multiplied by the same
fraction 1/N and then summed. In this case again, averaging is an additive
process.

In inclusive foreign aid, this approach can be used in both the discovery and
justification contexts. In the context of discovery, observing and hearing the
input of the target population can be a great way to begin assessing the
situation of a given population. Without a firm grasp of what might even
potentially constitute a good solution to a given population's problem,
participatory research can be purely exploratory, void of the goal of
justifying a particular knowledge claim.

On the other hand, participatory research can function as a valid approach in
the context of justification. Previously, the two levels of analysis for
assessing data for CBAs were outlined: the micro and macro levels. The macro
level approach includes metrics like poverty and income levels. However, what
should one do if these data aren't available? Similarly, how does one go about
obtaining such data? One must begin at the micro-level.

In the case of addressing the exclusion mechanisms, participatory research
offers an extremely valuable perspective on whether or not a policy is doing
good for PWDs; it asks the PWDs directly. This is a much more direct approach
than going through the MFI to see how a particular customer's balance is
changing and trying to interpret that in terms of the PWDs wellbeing
(completely disregarding how invasive the latter is). While serving a
justificatory purpose, if the proposed measure fails to do good for the PWD,
the aim of participatory research can then (in the same conversation) turn
into a more discovery oriented endeavor. It can begin to explore possible
explanations for why the proposed measure failed and give rise to new ideas of
possible exclusion mechanisms that may need to be addressed. So not only can
participatory research be employed in both contexts of discovery and
justification, it can be employed in both contexts simultaneously.


\paragraph{Link to knowledge claim}

\cite{kuhn2012structure} argued that the two contexts are not fundamental
aspects of the knowledge generation process. In his view, the discovery and
justification processes are not elementary components of logic and
epistemology, but rather essential parts of the theories for which they played
an important role in forming.

Kuhn explains how the concept of discovery is actually a fuzzy concept and
intertwined with theory. He explains this based on an example of oxygen.
There were multiple people who could be considered candidates for the
discovery of oxygen since they isolated it to some extent. They may have known
that they had produced and isolated something new, but they did not
necessarily know it was oxygen. Discovery and theory become intertwined when
one realizes that it is not necessarily the ``discoverer'' who proclaims
himself as the discoverer in the instant of discovery
    \footnote{Kuhn also argued that first discovering something is different
        than seeing something because it doesn't happen in an instant.
        Discovery involves both realizing that one has discovered something,
        and realizing what one has discovered, and this can take time.}.
On the contrary, it can be other people at a later point in time who come to a
consensus about who the title of discoverer. Discoveries can be disruptive to
the current theory, and the theory may need to be adjusted to incorporate this
new discovery. Accordingly, the discoverer and the discovery are named and
interpreted within the view of the current state of the theory. A similar
situation arises in the justification process. In justifying the theory, one
uses elements of the theory to justify it.

In inclusive foreign aid, one can clearly see how elements of the theory are
integral parts of the justification process. The justification process is not
merely a part of the theory nor is the theory merely a part of the
justification, but they are inextricably linked. In the example of inclusive
foreign aid, the metric of poverty is used to justify knowledge claims.
Originally poverty was defined simply as having income below a certain level
and a given knowledge claim was then justified with respect to this simplistic
definition. As the notion of poverty evolved and its multi-dimensional nature
came to be appreciated, certain knowledge claims that had been justified with
respect to the simplistic definition were no longer justified with respect to
the multi-dimensional definition. Nothing in the logic of the justification
process changed, but rather something prior to the logic: one of the premises
and definitions upon which the logical argumentation was based. So the context
of justification surely involves logical processes, but it also involves
certain givens, and it is these givens that are both part of the theory and
part of the justification process.


\newpage

\section{Transdisciplinarity in inclusive foreign aid}

\subsection{Transdisciplinarity explained}

The concept of transdisciplinarity is a subtopic in the field of
interdisciplinarity. To get an idea of exactly what transdisciplinarity is, it
is helpful to have an overview of some basic concepts in the field of
interdisciplinarity. \cite{klein2010taxonomy} provides a detailed overview of
the different types of interdisciplinarity.

\paragraph{Disciplinarity}

Interdisciplinarity stands in contrast to the standard disciplines commonly
found in the modern university. These include classic subjects like physics,
chemistry, economics, psychology, etc. Although seemingly set in stone in the
modern university, these disciplines evolved into their current form over the
course of the last 200 years \cite{weingart2010short}. As is usual with
historical processes, their evolution is not entirely predetermined and if it
were to happen again, there stands the chance that the disciplines would have
emerged in a different form. Given the arbitrariness involved, one can wonder
which other disciplines could have emerged given certain other circumstances.
One can envision a merging or fracturing of the various modern disciplines
that might be more directed at solving slightly or potentially even
substantially different problems. This leads us to the different types of
interdisciplinarity.

\paragraph{Multidisciplinarity}

The most rudimentary form of interdisciplinarity is known as
multidisciplinarity. It can be thought of as the first step on the way to
transdisciplinarity from traditional academic disciplines.
Multidisciplinarity involves the perspective of multiple disciplines. By
incorporating multiple disciplines, the multidisciplinary approach provides
access to more knowledge and methods than a traditional disciplinary approach,
but each discipline remains fully intact and separate. In other words, there
is a lack of integration of the different perspectives. Words like
juxtaposing, sequencing, and coordinating fit to the multidisciplinary
approach.

\paragraph{Interdisciplinarity}

A more advanced form of interdisciplinarity is known as interdisciplinarity.
The primary difference is the degree of integration of the individual
disciplinary perspectives that is achieved. An interdisciplinary approach can
involve borrowing the methods of one discipline to apply them to a problem in
a different discipline, however the degree to which this impacts the borrowing
discipline can vary. A stronger or more ``genuine'' form of
interdisciplinarity can be the result of theoretical integration. In this
sense of interdisciplinarity, one overarching theoretical framework can offer
insight into multiple disciplines, and findings in one discipline can
contribute to the problems and theories of another. Words like interacting,
integrating, focusing, blending, and linking aptly describe this approach.

\paragraph{Transdisciplinarity}

Finally we arrive at transdisciplinarity. Transdisciplinarity, in the sense
relevant to this work, focuses on the generation of knowledge directed at
solving real-world problems. This amounts to letting the real-world and its
inhabitants dictate which problems are important and in need of solving as
opposed to the traditional academic disciplinary structures. Because of the
potential divergence between the needs of a specific real-world problem and
the specialization of a traditional discipline, it is possible that the
knowledge and methods needed to address a given real-world problem are spread
across different disciplines. This is a similar concept to
interdisciplinarity as explained above, with the additional condition that
external (to academia) stakeholders also contribute to the knowledge
generation process. As a result, transdisciplinarity can be thought of as
interdisciplinarity with societal input. Words like transcending,
transgressing and transforming describe this approach

\paragraph{An instructive example}

\cite{latour1999circulating} details an expedition in which a botanist,
pedologist and geomorphologist investigate the boundary between a forest and
savannah. They wish to determine they dynamics at the interface; is the forest
or savannah advancing, or perhaps neither? Upon first glance, the botanist's
analysis leads to the conclusion that forest could either be advancing or
receding, and the pedologist's analysis leads to the conclusion that the
savannah is advancing. Merely comparing the two conclusions from the two
disciplines amounts to a multidisciplinary approach. However, their
contrasting conclusions are the reason for further investigation.

While the botanist and pedologist are taking measurements and discussing the
significance of the results, the geomorphologist ``adds her two cents to all
the conversations, allowing her expatriate colleagues to 'triangulate' their
judgments through hers'' \citep{latour1999circulating}. This exchange is
exemplary of an interdisciplinary approach. There are three disciplines
present, and each is interacting with and integrating the viewpoints of the
others. It differs from the multidisciplinary approach in that there is an
exchange between the individual disciplinary standpoints.

In order to see transdisciplinarity in action, there would have to be input
from external stakeholders. Latour mentions the significance that findings of
this investigation might have for potential investors, but this is not enough.
In order to display transdisciplinarity, the investors need to play a role in
the knowledge generation process; their interest is not enough, their input is
necessary.

\subsection{Inclusive foreign aid as a transdisciplinary field}

Inclusive foreign aid is a transdisciplinary field because it addresses a
societally relevant problem with the input of stakeholders and integrates
findings and methodologies from various disciplines.

\paragraph{Extra-academic genesis of the problem and stakeholder input}

The genesis of this problem clearly does not lie in any of the traditional
disciplines. The entire concept of foreign aid is focused on solving
real-world problems, and as a result there are stakeholders involved.

The stakeholders in any endeavor relevant to inclusive foreign aid are those
funding, planning, implementing and benefiting from any proposed measures.
These categories are not necessarily mutually exclusive. Ensuring that those
receiving the aid play a role in the knowledge generation process is in line
with the slogan ``Nothing about us without us'' found in the disability
studies literature \citep{pfeiffer2000disability}. The means by which the
recipients play a role in the knowledge generation process is known as
participatory research, which is explained in \autoref{part}.

\paragraph{What types of things are drawn on from the different disciplines?}

Inclusive foreign aid includes aspects from a variety of different disciplines
including disability studies, finance, economics, law, education, technology,
medicine, psychology, sociology, bionics and engineering.

As explained above, there is a difference between calling upon the knowledge and
methodologies of those disciplines in a merely juxtapositional manner and
doing so in a more integrative manner (i.e. multidisciplinarity vs.
interdisciplinarity).

Inclusive foreign aid borrows many concepts and methods from disability
studies. Some of these things include the paradigms and models of disability
and the survey methods. The issue of how to effectively conduct surveys is a
good example of borrowing a methodology from one domain and adapting it to fit
the current domain.

Inclusive foreign aid also borrows concepts in economics of poverty,
inequality, asset index and economic mobility. Viewing disability through the
lens of inequality allows one to determine its influence on economic welfare.
For example, by measuring the asset index \footnote{An asset index measures
wealth not merely by income, but also considers basic assets like access to
basic sanitation, running water, indoor plumbing, etc.} in different countries
and across income groups, functionally limited people were found to be two to
four times more likely to be in the lowest income quintile than in the highest
income quintile \citep{mitra2018disability}.

As a final example, inclusive foreign aid borrows concepts from the field of
technology in the way of prosthetics and other assistive technologies
\citep{roulstone2016disability}. These technologies, in very simple cases,
have the potential eliminate a disability. In surveys of Ethiopia, Malawi,
Tanzania and Uganda, the most common disability for those with both severe and
moderate functional difficulties was sight-related
\citep{mitra2018disability}. It was also found that assistive technologies
were largely unused due to either cost of availability issues. Simple measures
such as glasses would be sufficient to greatly reduce the number of people who
identify themselves as disabled. This insight sheds light on the validity of
two paradigms in inclusive foreign aid and disability studies, which will be
explained in more detail in \autoref{models}.



%%%%%%%%%%%%%%%%%%%%%%
%%%%%%%%%%%%%%%%%%%%%%
%%
%% Appendix
%%
%%%%%%%%%%%%%%%%%%%%%%
%%%%%%%%%%%%%%%%%%%%%%

\newpage
\begin{appendices}
\section{Quasi-objects}

\cite{latour2012we} outlines the idea of a quasi-object. He presents a problem
with the traditional subject/society and object/nature dichotomy. There are
two faulty beliefs refuted by contradictory denunciations when it comes to the
interaction between objects and subjects.
%
\begin{itemize}
    \item[] Belief 1: Objects have an intrinsic value (i.e. money, gods,
        status, etc).

    \item[] Denunciation 1: Those objects are intrinsically meaningless and
        serve merely as blank slates onto which society projects its values.
\end{itemize}
%
In this view of things, the objects are weak and worthless, and society is
powerful enough to be capable of making anything out of them. 
%
\begin{itemize}
    \item[] Belief 2: Subjects have free will and the power to do whatever it
        is they like.

    \item[] Denunciation 2: There are natural constraints on subjects that
        place limitations on what they can do. 
\end{itemize}
%
In this view of things, subjects are weak over the influence of an
overwhelmingly powerful nature. 

These two denunciations are incompatible because objects are
meaningless in the first but powerful enough to influence the fate of humans
in the second, while subjects are capable of shaping the natural world to an
arbitrary degree in the first and powerless in the second.

This contradiction can be addressed by positing a dualism in both the objects
and subjects: objects have a hard ``pole'' that is capable of shaping and
limiting behavior of the soft pole of subjects, and also a soft pole upon
which the hard pole of culture can project its values. Said differently,
subjects have a hard pole that is capable of projecting and giving value to
the soft pole of objects, and a soft pole that is shaped and limited by the
hard pole of the objects. The problem with this is that the list of things
that make up the hard and soft parts are somewhat arbitrary. The hard parts of
nature are whatever of the natural sciences one believes, and the hard parts
of society are whatever of the social sciences one believes.

When social scientists deployed the same arguments that were used to debunk
the soft components of nature on the hard components of nature (the sciences
themselves that were foundational aspects of their beliefs) by seeing them as
the products of the society's interests and requirements, the whole dualistic
paradigm collapsed. Once it was deemed absurd that society constructed all of
the scientific facts out of its own self-interest, it was also seen as much
less plausible that these soft facts could just as easily be dismissed in the
same manner.

Quasi-objects are offered as the solution to the problem that not all objects
and subjects are only either indefinitely malleable or infinitely powerful
i.e. ``soft'' or ``hard.'' To take a direct quote from Latour,

\begin{quote}
Quasi-objects are much more social, much more fabricated, much more
collective than the 'hard' parts of nature, but they are in no way the
arbitrary receptacles of a full-fledged society. On the other hand they are
much more real, nonhuman and objective than those shapeless screens on which
society – for unknown reasons – needed to be 'projected.' \citep{latour2012we}
\end{quote}

The proposition that will be expanded on below is that prosthetics and other
augmentations that enable PWDs to become more ``normal'' are quasi-objects.
Additionally that this has the potential to unify two different paradigms of
disability studies.

\subsection{Prosthetics as quasi-objects}

The connection between a stump leg and a quasi-object was previously made by
\cite{bertram2018bestial}. The proposition relevant for this discussion is
that prosthetics are quasi-objects. Seen through the lens of the
subject/object dichotomy, a prosthetic is clearly not a subject because it is
an inanimate object. It is true that a prosthetic itself is a nonhuman,
nonliving, inorganic object, but seen through the hard/soft pole dichotomy
that Latour explains, is a prosthetic on the list of hard or soft traits?

In one sense, it belongs to the hard side because it is able to impact the
experience of subjects. Prosthetics allow for very real changes in the lives
of PWDs. Examples include enabling leg amputees to run, the deaf to hear, the
blind to see, etc. A prosthetic is not an intrinsically meaningless object
upon which society merely projects its values because it offers value to a PWD
independent of the society and culture. Thus the first denunciation is
debunked.

In another sense, it belongs to the soft side because it is the object of
human creation. Humans have designed this to suit their needs, and further
modifications can be made to reflect advancements in technology, ergonomics,
and user satisfaction. At any rate, a prosthetic cannot be said to be an all
powerful object over which humans have no control. Thus the second
denunciation above is debunked.

Once the two denunciations have been found to not apply, i.e. that this object
is neither exclusively on the hard pole nor the soft pole, we find ourselves
in the territory of quasi-objects.

\end{appendices}






\cleardoublepage
\bibliography{mybib}
\bibliographystyle{plainnat}
%\bibliographystyle{ieeetr}
%\bibliographystyle{detailed}

%\begin{thebibliography}{AAA}
%\bibitem{mypaper} \textsc{First, J.} and \textsc{Second, J.} (2010), ``An interesting paper'', \emph{A Famous Journal}, 1, pp.~1--11.
%\bibitem{mybook} \textsc{First, J.} and \textsc{Second, J.} (2011), \emph{A great book}, Address: Publisher.
%\end{thebibliography}

\end{document}
%%%%%%%%%%%%%%%%%%%%%%%%%%%%%%%%%%%%%%%%%%%%%%%%%%%%%%%%%%%%%%%%%%%%%%%%%%%%
\par

